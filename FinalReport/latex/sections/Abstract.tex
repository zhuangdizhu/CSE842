% !TEX root = /Users/zhuzhuangdi/Desktop/MSUCourses/NaturalLanguageProcessing842/CSE842/FinalReport/latex/final.tex

\begin{abstract}
In this project, we explore the application of Recursive Neural Network (RNN) models on the sentiment analysis task with Twitter.
%
Twitter is a dominating social media that allows users to post 140-character long micro blogs called tweets, which usually convey large amount of information that can be further used for opinion mining on important social events. 
%
Different from prior work using bag-of-words models, we plan to evaluate how deep learning models can be used to improve the accuracy of sentence-level sentiment analysis in micro-blogging domain. 
%
We use word-to-vector approaches to map each word in a tweet into a vector representation which capture the semantic relevance among different words, and use the vector representation as inputs to training our RNN classifier.
%
Our RNN approach outperforms traditional bag-of-words based classifiers by over 20\% percent accuracy.
%
It also outperforms a Convolution Neural Network based approach by over \% percent accuracy, with a faster convergence rate.
%
Our model can also be scaled to solving multi-class classification problems.

% We semantically compose vector-space representations for tweets to capture their parse structure.
%Especially, we will use the treebank provided by Stanford to generate a parse tree for each tweet, and use it as the input to the RNN model. 
%

%

\end{abstract}
%\keywords{Semantic Analysis; Twitter; Micro-blogging; Parse Tree;  Deep Learning }